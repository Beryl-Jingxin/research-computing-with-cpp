\section{Lecture 1: C++ for Research}\label{lecture-1-c-for-research}

\subsection{Course Overview}\label{course-overview}

\subsubsection{Part 1}\label{part-1}

\begin{itemize}
\itemsep1pt\parskip0pt\parsep0pt
\item
  Using C++ in research
\item
  Better C++

  \begin{itemize}
  \itemsep1pt\parskip0pt\parsep0pt
  \item
    Reliable
  \item
    Reproducible
  \item
    Good science
  \item
    Libraries
  \end{itemize}
\end{itemize}

\subsubsection{Part 2}\label{part-2}

\begin{itemize}
\itemsep1pt\parskip0pt\parsep0pt
\item
  HPC concepts
\item
  Shared memory parallelism - \href{http://www.openmp.org}{OpenMP}
\item
  Distributed memory parallelism - \href{http://www.open-mpi.org}{MPI}
\end{itemize}

\subsubsection{Course Aims}\label{course-aims}

\begin{itemize}
\itemsep1pt\parskip0pt\parsep0pt
\item
  Teach how to do research with C++
\item
  Optimise your research output
\item
  A taster for various technologies
\item
  Not just C++ syntax, Google/Compiler could tell you that!
\end{itemize}

\subsubsection{Pre-requisites}\label{pre-requisites}

\begin{itemize}
\itemsep1pt\parskip0pt\parsep0pt
\item
  Use of command line (Unix) shell
\item
  You are already doing some C++
\item
  You are familiar with your compiler
\item
  (Maybe) You are happy with the concept of classes
\item
  (Maybe) You know C++ up to templates?
\item
  You are familiar with development eg. version control

  \begin{itemize}
  \itemsep1pt\parskip0pt\parsep0pt
  \item
    Git: \url{https://git-scm.com/}
  \end{itemize}
\end{itemize}

\subsubsection{Course Notes}\label{course-notes}

\begin{itemize}
\itemsep1pt\parskip0pt\parsep0pt
\item
  Revise some software Engineering:
  \href{http://github-pages.ucl.ac.uk/rsd-engineeringcourse/}{MPHY0021}
\item
  Register with Moodle: \href{https://moodle.ucl.ac.uk/}{PHAS0100}

  \begin{itemize}
  \itemsep1pt\parskip0pt\parsep0pt
  \item
    contact lecturer for key to self-register
  \item
    guest key to look around is ``996135''
  \end{itemize}
\item
  Online notes:
  \href{http://rits.github-pages.ucl.ac.uk/research-computing-with-cpp/}{PHAS0100}
\end{itemize}

\subsubsection{Course Assessment}\label{course-assessment}

\begin{itemize}
\itemsep1pt\parskip0pt\parsep0pt
\item
  2 pieces coursework - 40 hours each

  \begin{itemize}
  \itemsep1pt\parskip0pt\parsep0pt
  \item
    See assessment section for details
  \end{itemize}
\end{itemize}

\subsubsection{Course Community}\label{course-community}

\begin{itemize}
\itemsep1pt\parskip0pt\parsep0pt
\item
  UCL Research Programming Hub:
  \href{http://research-programming.ucl.ac.uk/}{http://research-programming.ucl.ac.uk}
\item
  Slack:
  \href{https://ucl-programming-hub.slack.com/}{https://ucl-programming-hub.slack.com}
\end{itemize}

\subsection{Lecture 1: C++ In Research}\label{lecture-1-c-in-research}

\subsubsection{Problems In Research}\label{problems-in-research}

\begin{itemize}
\itemsep1pt\parskip0pt\parsep0pt
\item
  Poor quality software
\item
  Excuses

  \begin{itemize}
  \itemsep1pt\parskip0pt\parsep0pt
  \item
    I'm not a software engineer
  \item
    I don't have time
  \item
    It's just a prototype
  \item
    I'm unsure of my code (scared to share)
  \end{itemize}
\end{itemize}

\subsubsection{C++ Disadvantages}\label{c-disadvantages}

Some people say:

\begin{itemize}
\itemsep1pt\parskip0pt\parsep0pt
\item
  Compiled language

  \begin{itemize}
  \itemsep1pt\parskip0pt\parsep0pt
  \item
    (compiler versions, libraries, platform specific etc)
  \end{itemize}
\item
  Perceived as difficult, error prone, wordy, unfriendly syntax
\item
  Result: It's more trouble than its worth?
\end{itemize}

\subsubsection{C++ Advantages}\label{c-advantages}

\begin{itemize}
\itemsep1pt\parskip0pt\parsep0pt
\item
  Fast, code compiled to machine code
\item
  Stable, evolving standard, powerful notation, improving
\item
  Lots of libraries, Boost, Qt, VTK, ITK etc.
\item
  Nice integration with CUDA, OpenACC, OpenCL, OpenMP, OpenMPI
\item
  Result: Good reasons to use it, or you may \emph{have} to use it
\end{itemize}

\subsubsection{Research Programming}\label{research-programming}

\begin{itemize}
\itemsep1pt\parskip0pt\parsep0pt
\item
  Software is always expensive

  \begin{itemize}
  \itemsep1pt\parskip0pt\parsep0pt
  \item
    Famous Book:
    \href{http://www.amazon.co.uk/Mythical-Man-month-Essays-Software-Engineering/dp/0201835959/ref=sr_1_1?ie=UTF8\&qid=1452507457\&sr=8-1\&keywords=mythical+man+month}{Mythical
    Man Month}
  \end{itemize}
\item
  Research programming is different to product development:

  \begin{itemize}
  \itemsep1pt\parskip0pt\parsep0pt
  \item
    What is the end product?
  \end{itemize}
\end{itemize}

\subsubsection{Development Methodology?}\label{development-methodology}

\begin{itemize}
\itemsep1pt\parskip0pt\parsep0pt
\item
  Will software engineering methods help?

  \begin{itemize}
  \itemsep1pt\parskip0pt\parsep0pt
  \item
    \href{https://en.wikipedia.org/wiki/Waterfall_model}{Waterfall}
  \item
    \href{https://en.wikipedia.org/wiki/Agile_software_development}{Agile}
  \end{itemize}
\item
  At the `concept discovery' stage, probably too early to talk about
  product development
\end{itemize}

\subsubsection{Approach}\label{approach}

\begin{itemize}
\itemsep1pt\parskip0pt\parsep0pt
\item
  What am I trying to achieve?
\item
  How do I maximise my research output?
\item
  What is the best pathway to success?
\item
  How do I de-risk (get results, meet deadlines) my research?
\item
  Software is an important part of scientific reproducibility,
  authorship, credibility.
\end{itemize}

\subsubsection{1. Types of Code}\label{types-of-code}

\begin{itemize}
\itemsep1pt\parskip0pt\parsep0pt
\item
  What are you trying to achieve?
\item
  Divide code:

  \begin{itemize}
  \itemsep1pt\parskip0pt\parsep0pt
  \item
    Your algorithm
  \item
    Testing code
  \item
    Data analysis
  \item
    User Interface
  \item
    Glue code
  \item
    Deployment code
  \item
    Scientific output (a paper)
  \end{itemize}
\end{itemize}

Question: What type of code is C++ good for? Question: Should all code
be in C++?

\subsubsection{2. Maximise Your Value}\label{maximise-your-value}

\begin{itemize}
\itemsep1pt\parskip0pt\parsep0pt
\item
  Developer time is expensive
\item
  Your brain is your asset
\item
  Write as little code as possible
\item
  Focus tightly on your hypothesis
\item
  Write the minimum code that produces a paper
\end{itemize}

Don't fall into the trap ``Hey, I'll write a framework for that''

\subsubsection{3. Ask Advice}\label{ask-advice}

\begin{itemize}
\itemsep1pt\parskip0pt\parsep0pt
\item
  Before contemplating a new piece of software

  \begin{itemize}
  \itemsep1pt\parskip0pt\parsep0pt
  \item
    Ask advice - \href{https://ucl-programming-hub.slack.com/}{Slack
    Channel}
  \item
    Review libraries and use them.
  \item
    Check libraries are suitable, and sustainable.
  \item
    Read
    \href{http://development.rc.ucl.ac.uk/training/engineering/ch04packaging/01Libraries.html}{Libraries}
    section from
    \href{http://github-pages.ucl.ac.uk/rsd-engineeringcourse/}{Software
    Engineering} course
  \item
    Ask about best practices
  \end{itemize}
\end{itemize}

\subsubsection{Debunking The Excuses}\label{debunking-the-excuses}

\begin{itemize}
\itemsep1pt\parskip0pt\parsep0pt
\item
  I'm not a software engineer

  \begin{itemize}
  \itemsep1pt\parskip0pt\parsep0pt
  \item
    Learn effective, minimal tools
  \end{itemize}
\item
  I don't have time

  \begin{itemize}
  \itemsep1pt\parskip0pt\parsep0pt
  \item
    Unit testing to save time
  \item
    Choose your battles/languages wisely
  \end{itemize}
\item
  I'm unsure of my code

  \begin{itemize}
  \itemsep1pt\parskip0pt\parsep0pt
  \item
    Share, collaborate
  \end{itemize}
\end{itemize}

\subsubsection{What Isn't This Course?}\label{what-isnt-this-course}

We are NOT suggesting that:

\begin{itemize}
\itemsep1pt\parskip0pt\parsep0pt
\item
  C++ is the solution to all problems.
\item
  You should write all parts of your code in C++.
\end{itemize}

\subsubsection{What Is This Course?}\label{what-is-this-course}

We aim to:

\begin{itemize}
\itemsep1pt\parskip0pt\parsep0pt
\item
  Improve your C++ (and associated technologies).
\item
  Introduction to High Performance Computing (HPC).
\end{itemize}

So that:

\begin{itemize}
\itemsep1pt\parskip0pt\parsep0pt
\item
  Apply it to research in a pragmatic fashion.
\item
  You use the right tool for the job.
\end{itemize}

\subsection{Git}\label{git}

\subsubsection{Git Introduction}\label{git-introduction}

\begin{itemize}
\itemsep1pt\parskip0pt\parsep0pt
\item
  This is a practical course
\item
  We will use git for version control
\item
  Submit git repository for coursework
\item
  Here we provide a very minimal introduction
\end{itemize}

\subsubsection{Git Resources}\label{git-resources}

\begin{itemize}
\itemsep1pt\parskip0pt\parsep0pt
\item
  Complete beginner - \href{https://try.github.io}{Try Git}
\item
  \href{https://git-scm.com/book/en/v2}{Git book by Scott Chacon}
\item
  \href{http://github-pages.ucl.ac.uk/rsd-engineeringcourse/ch02git/}{Git
  section} of MPHYG001
\item
  \href{https://github.com/UCL/rsd-engineeringcourse}{MPHYG001 repo}
\end{itemize}

\subsubsection{Git Walk Through - 1}\label{git-walk-through---1}

(demo on command line)

\begin{itemize}
\itemsep1pt\parskip0pt\parsep0pt
\item
  Creating your own repo:

  \begin{itemize}
  \itemsep1pt\parskip0pt\parsep0pt
  \item
    git init
  \item
    git add
  \item
    git commit
  \item
    git log
  \item
    git status
  \item
    git remote add
  \item
    git push
  \end{itemize}
\end{itemize}

\subsubsection{Git Walk Through - 2}\label{git-walk-through---2}

(demo on command line)

\begin{itemize}
\itemsep1pt\parskip0pt\parsep0pt
\item
  Working on existing repo:

  \begin{itemize}
  \itemsep1pt\parskip0pt\parsep0pt
  \item
    git clone
  \item
    git add
  \item
    git commit
  \item
    git log
  \item
    git status\\
  \item
    git push
  \item
    git pull
  \end{itemize}
\end{itemize}

\subsubsection{Git Walk Through - 3}\label{git-walk-through---3}

\begin{itemize}
\itemsep1pt\parskip0pt\parsep0pt
\item
  Cloning or forking:

  \begin{itemize}
  \itemsep1pt\parskip0pt\parsep0pt
  \item
    If you have write permission to a repo: clone it, and make
    modifications
  \item
    If you don't: fork it, to make your own version, then clone that and
    make modifications.
  \end{itemize}
\end{itemize}

\subsubsection{Homework - 1}\label{homework---1}

\begin{itemize}
\itemsep1pt\parskip0pt\parsep0pt
\item
  Register Github
\item
  Register for private repos - free for academia.
\item
  Find project of interest - try cloning it, make edits, can you push?
\item
  Find project of interest - try forking it, make edits, can you push?
\end{itemize}

\subsection{CMake}\label{cmake}

\subsubsection{Ever worked in Industry?}\label{ever-worked-in-industry}

\begin{itemize}
\itemsep1pt\parskip0pt\parsep0pt
\item
  (0-3yrs) Junior Developer - given environment, team support
\item
  (4-6yrs) Senior Developer - given environment, leading team
\item
  (7+ years) Architect - chose tools, environment, design code
\item
  Only cross-platform if product/business demands it
\item
  All developers told to use the given platform no choice
\end{itemize}

\subsubsection{Ever worked in Research?}\label{ever-worked-in-research}

\begin{itemize}
\itemsep1pt\parskip0pt\parsep0pt
\item
  All prototyping, no scope
\item
  Start from scratch, little support
\item
  No end product, no nice examples
\item
  Cutting edge use of maths/science/technology
\item
  Share with others on other platforms
\item
  Develop on Windows, run on cluster (Linux)
\end{itemize}

\subsubsection{Research Software Engineering
Dilemma}\label{research-software-engineering-dilemma}

\begin{itemize}
\itemsep1pt\parskip0pt\parsep0pt
\item
  Comparing Research with Industry, in Research you have:

  \begin{itemize}
  \itemsep1pt\parskip0pt\parsep0pt
  \item
    Least experienced developers
  \item
    with the least support
  \item
    developing cross-platform
  \item
    No clear specification or scope
  \end{itemize}
\item
  Struggle of C++ is often not the language its the environment
\end{itemize}

\subsubsection{Build Environment}\label{build-environment}

\begin{itemize}
\itemsep1pt\parskip0pt\parsep0pt
\item
  Windows: Visual Studio solution files
\item
  Linux: Makefiles
\item
  Mac: XCode projects / Makefiles
\end{itemize}

Question: How was your last project built?

\subsubsection{CMake Introduction}\label{cmake-introduction}

\begin{itemize}
\itemsep1pt\parskip0pt\parsep0pt
\item
  This is a practical course
\item
  We will use CMake as a build tool
\item
  CMake produces

  \begin{itemize}
  \itemsep1pt\parskip0pt\parsep0pt
  \item
    Windows: Visual Studio project files
  \item
    Linux: Make files
  \item
    Mac: XCode projects, Make files
  \end{itemize}
\item
  So you write 1 build language (CMake) and run on multi-platform.
\item
  This course will provide most CMake code and boiler plate code for
  you, so we can focus more on C++. But you are expected to google CMake
  issues and work with CMake.
\end{itemize}

\subsubsection{CMake Usage Linux/Mac}\label{cmake-usage-linuxmac}

Demo an ``out-of-source'' build

\begin{Shaded}
\begin{Highlighting}[]
\KeywordTok{cd} \NormalTok{~/build}
\KeywordTok{git} \NormalTok{clone https://github.com/MattClarkson/CMakeHelloWorld}
\KeywordTok{mkdir} \NormalTok{CMakeHelloWorld-build}
\KeywordTok{cd} \NormalTok{CMakeHelloWorld-build}
\KeywordTok{ccmake} \NormalTok{../CMakeHelloWorld}
\KeywordTok{make}
\end{Highlighting}
\end{Shaded}

\subsubsection{Homework - 2}\label{homework---2}

\begin{itemize}
\itemsep1pt\parskip0pt\parsep0pt
\item
  Build https://github.com/MattClarkson/CMakeHelloWorld.git
\item
  Ensure you do ``out-of-source'' builds
\item
  Use CMake to configure separate Debug and Release versions
\item
  Add code to hello.cpp:

  \begin{itemize}
  \itemsep1pt\parskip0pt\parsep0pt
  \item
    On Linux/Mac re-compile just using make
  \end{itemize}
\end{itemize}

\subsubsection{Homework - 3}\label{homework---3}

\begin{itemize}
\itemsep1pt\parskip0pt\parsep0pt
\item
  Build https://github.com/MattClarkson/CMakeHelloWorld.git
\item
  Exit all code editors
\item
  Rename hello.cpp
\item
  Change CMakeLists.txt accordingly
\item
  Notice: The executable name and .cpp file name can be different
\item
  In your build folder, just try rebuilding.
\item
  You should see that CMake is re-triggered, so you get a cmake/compile
  cycle.
\end{itemize}

\subsection{CMake Basics}\label{cmake-basics}

\subsubsection{Compiling Basics}\label{compiling-basics}

Question: How does a compiler work?

\subsubsection{How does a Compiler
Work?}\label{how-does-a-compiler-work}

Question: How does a compiler work?

\begin{itemize}
\itemsep1pt\parskip0pt\parsep0pt
\item
  (Don't quote any of this in a compiler theory course!)
\item
  Preprocessing .cpp/.cxx into pure source files
\item
  Source files compiled, one by one into .o/.obj
\item
  executable compiled to .o/.obj
\item
  executable linked against all .o, and all libraries
\end{itemize}

That's what you are trying to describe in CMake.

\subsubsection{CMake - Directory
Structure}\label{cmake---directory-structure}

\begin{itemize}
\itemsep1pt\parskip0pt\parsep0pt
\item
  CMake starts with top-level CMakeLists.txt
\item
  CMakeLists.txt is read top-to-bottom
\item
  All CMake code goes in CMakeLists.txt or files included from a
  CMakeLists.txt
\item
  You can sub-divide your code into separate folders.
\item
  If you \texttt{add\_subdirectory}, CMake will go into that directory
  and start to process the CMakeLists.txt therein. Once finished it will
  exit, go back to directory above and continue where it left off.
\item
  e.g.~top level CMakeLists.txt
\end{itemize}

\begin{Shaded}
\begin{Highlighting}[]
\KeywordTok{project}\NormalTok{(MYPROJECT }\OtherTok{VERSION} \NormalTok{0.0.0)}
\KeywordTok{add_subdirectory}\NormalTok{(Code)}
\KeywordTok{if}\NormalTok{(BUILD_TESTING)}
  \KeywordTok{set}\NormalTok{()}
  \KeywordTok{include_directories}\NormalTok{()}
  \KeywordTok{add_subdirectory}\NormalTok{(Testing)}
\KeywordTok{endif}\NormalTok{()}
\end{Highlighting}
\end{Shaded}

\subsubsection{CMake - Define Targets}\label{cmake---define-targets}

\begin{itemize}
\itemsep1pt\parskip0pt\parsep0pt
\item
  Describe a target, e.g.~Library, Application, Plugin
\end{itemize}

\begin{Shaded}
\begin{Highlighting}[]
\KeywordTok{add_executable}\NormalTok{(hello hello.cpp)}
\end{Highlighting}
\end{Shaded}

\begin{itemize}
\itemsep1pt\parskip0pt\parsep0pt
\item
  Note: You don't write compile commands
\item
  You tell CMake what things need compiling to build a given target.
  CMake works out the compile commands!
\end{itemize}

\subsubsection{CMake - Order Dependent}\label{cmake---order-dependent}

\begin{itemize}
\itemsep1pt\parskip0pt\parsep0pt
\item
  You can't say ``build Y and link to X'' if X not defined
\item
  So, imagine in a larger project
\end{itemize}

\begin{Shaded}
\begin{Highlighting}[]
\KeywordTok{add_library}\NormalTok{(libA a.cpp b.cpp c.cpp)}
\KeywordTok{add_library}\NormalTok{(libZ x.cpp y.cpp z.cpp)}
\KeywordTok{target_link_libraries}\NormalTok{(libZ libA)}
\KeywordTok{add_executable}\NormalTok{(myAlgorithm algo.cpp) }\CommentTok{# contains main()}
\KeywordTok{target_link_libraries}\NormalTok{(myAlgorithm libA libZ }\DecValTok{$\{THIRD_PARTY_LIBS\}}\NormalTok{)}
\end{Highlighting}
\end{Shaded}

\begin{itemize}
\itemsep1pt\parskip0pt\parsep0pt
\item
  So, logically, its a big, ordered set of build commands.
\end{itemize}

\subsubsection{Homework - 4}\label{homework---4}

\begin{itemize}
\itemsep1pt\parskip0pt\parsep0pt
\item
  Build https://github.com/MattClarkson/CMakeLibraryAndApp.git
\item
  Look through .cpp/.h code. Ask questions if you don't understand it.
\item
  What is an ``include guard''?
\item
  What is a namespace?
\item
  Look at .travis.yml and appveyor.yml - cross platform testing, free
  for open-source
\item
  Look at myApp.cpp, does it make sense?
\item
  Look at CMakeLists.txt, does it make sense?
\item
  Look for examples on the web, e.g.
  \href{https://lorensen.github.io/VTKExamples/site/Cxx/GeometricObjects/Cone/}{VTK}
\end{itemize}

\subsection{Intermediate CMake}\label{intermediate-cmake}

\subsubsection{What's next?}\label{whats-next}

\begin{itemize}
\itemsep1pt\parskip0pt\parsep0pt
\item
  Most people learn CMake by pasting snippets from around the web
\item
  As project gets larger, its more complex
\item
  Researchers tend to just ``stick with what they have.''
\item
  i.e.~just keep piling more code into the same file.
\item
  Want to show you a reasonable template project.
\end{itemize}

\subsubsection{Homework - 5}\label{homework---5}

\begin{itemize}
\itemsep1pt\parskip0pt\parsep0pt
\item
  Build https://github.com/MattClarkson/CMakeCatch2.git
\item
  If open-source, use travis and appveyor from day 1.
\item
  We will go through top-level CMakeLists.txt in class.
\item
  See separate \texttt{Code} and \texttt{Testing} folders
\item
  Separate \texttt{Lib} and \texttt{CommandLineApps} and
  \texttt{3rdParty}
\item
  You should focus on

  \begin{itemize}
  \itemsep1pt\parskip0pt\parsep0pt
  \item
    Write a good library
  \item
    Unit test it
  \item
    Then it can be called from command line, wrapped in Python, used via
    GUI.
  \end{itemize}
\end{itemize}

\subsubsection{Homework - 6}\label{homework---6}

\begin{itemize}
\itemsep1pt\parskip0pt\parsep0pt
\item
  Try renaming stuff to create a library of your choice.
\item
  Create a github account, github repo, Appveyor and Travis account
\item
  Try to get your code running on 3 platforms
\item
  If you can, consider using this repo for your research
\item
  Discuss

  \begin{itemize}
  \itemsep1pt\parskip0pt\parsep0pt
  \item
    Debug / Release builds
  \item
    Static versus Dynamic
  \item
    declspec import/export
  \item
    Issues with running command line app? Windows/Linux/Mac
  \end{itemize}
\end{itemize}

\subsubsection{Looking forward}\label{looking-forward}

In the remainder of this course we cover

\begin{itemize}
\itemsep1pt\parskip0pt\parsep0pt
\item
  Some compiler options
\item
  Using libraries
\item
  Including libraries in CMake
\item
  Unit testing
\item
  i.e.~How to put together a C++ project
\item
  in addition to actual C++ and HPC
\end{itemize}

\subsection{Unit Testing}\label{unit-testing}

\subsubsection{What is Unit Testing?}\label{what-is-unit-testing}

At a high level

\begin{itemize}
\itemsep1pt\parskip0pt\parsep0pt
\item
  Way of testing code.
\item
  Unit

  \begin{itemize}
  \itemsep1pt\parskip0pt\parsep0pt
  \item
    Smallest `atomic' chunk of code
  \item
    i.e.~Function, could be a Class
  \end{itemize}
\item
  See also:

  \begin{itemize}
  \itemsep1pt\parskip0pt\parsep0pt
  \item
    Integration/System Testing
  \item
    Regression Testing
  \item
    User Acceptance Testing
  \end{itemize}
\end{itemize}

\subsubsection{Benefits of Unit
Testing?}\label{benefits-of-unit-testing}

\begin{itemize}
\itemsep1pt\parskip0pt\parsep0pt
\item
  Certainty of correctness
\item
  (Scientific Rigour)
\item
  Influences and improves design
\item
  Confidence to refactor, improve
\end{itemize}

\subsubsection{Drawbacks for Unit
Testing?}\label{drawbacks-for-unit-testing}

\begin{itemize}
\itemsep1pt\parskip0pt\parsep0pt
\item
  Don't know how

  \begin{itemize}
  \itemsep1pt\parskip0pt\parsep0pt
  \item
    This course will help
  \end{itemize}
\item
  Takes too much time

  \begin{itemize}
  \itemsep1pt\parskip0pt\parsep0pt
  \item
    Really?
  \item
    IT SAVES TIME in the long run
  \end{itemize}
\end{itemize}

\subsubsection{Unit Testing Frameworks}\label{unit-testing-frameworks}

Generally, all very similar

\begin{itemize}
\itemsep1pt\parskip0pt\parsep0pt
\item
  JUnit (Java), NUnit (.net?), CppUnit, phpUnit,
\item
  Basically

  \begin{itemize}
  \itemsep1pt\parskip0pt\parsep0pt
  \item
    Macros (C++), methods (Java) to test conditions
  \item
    Macros (C++), reflection (Java) to run/discover tests
  \item
    Ways of looking at results.

    \begin{itemize}
    \itemsep1pt\parskip0pt\parsep0pt
    \item
      Java/Eclipse: Integrated with IDE
    \item
      Log file or standard output
    \end{itemize}
  \end{itemize}
\end{itemize}

\subsection{Unit Testing Example}\label{unit-testing-example}

\subsubsection{How To Start}\label{how-to-start}

We discuss

\begin{itemize}
\itemsep1pt\parskip0pt\parsep0pt
\item
  Basic Example
\item
  Some tips
\end{itemize}

Then its down to the developer/artist.

\subsubsection{C++ Frameworks}\label{c-frameworks}

To Consider:

\begin{itemize}
\itemsep1pt\parskip0pt\parsep0pt
\item
  \href{https://github.com/philsquared/Catch}{Catch}
\item
  \href{https://code.google.com/p/googletest/}{GoogleTest}
\item
  \href{http://qt-project.org/doc/qt-4.8/qtestlib-manual.html}{QTestLib}
\item
  \href{http://www.boost.org/doc/libs/1_57_0/libs/test/doc/html/index.html}{BoostTest}
\item
  \href{http://cpptest.sourceforge.net/}{CppTest}
\item
  \href{http://sourceforge.net/projects/cppunit/}{CppUnit}
\end{itemize}

\subsubsection{Worked Example}\label{worked-example}

\begin{itemize}
\itemsep1pt\parskip0pt\parsep0pt
\item
  Borrowed from

  \begin{itemize}
  \itemsep1pt\parskip0pt\parsep0pt
  \item
    \href{https://github.com/philsquared/Catch/blob/master/docs/tutorial.md}{Catch
    Tutorial}
  \item
    and
    \href{https://code.google.com/p/googletest/wiki/V1_7_Primer}{Googletest
    Primer}
  \end{itemize}
\item
  We use \href{https://github.com/philsquared/Catch}{Catch}, so notes
  are compilable
\item
  But the concepts are the same
\end{itemize}

\subsubsection{Code}\label{code}

To keep it simple for now we do this in one file:

\begin{Shaded}
\begin{Highlighting}[]
\OtherTok{#define CATCH_CONFIG_MAIN  }\CommentTok{// This tells Catch to provide a main() - only do this in one cpp file}
\OtherTok{#include "../catch/catch.hpp"}

\DataTypeTok{unsigned} \DataTypeTok{int} \NormalTok{Factorial( }\DataTypeTok{unsigned} \DataTypeTok{int} \NormalTok{number ) \{}
    \KeywordTok{return} \NormalTok{number <= }\DecValTok{1} \NormalTok{? number : Factorial(number}\DecValTok{-1}\NormalTok{)*number;}
\NormalTok{\}}

\NormalTok{TEST_CASE( }\StringTok{"Factorials are computed"}\NormalTok{, }\StringTok{"[factorial]"} \NormalTok{) \{}
    \NormalTok{REQUIRE( Factorial(}\DecValTok{1}\NormalTok{) == }\DecValTok{1} \NormalTok{);}
    \NormalTok{REQUIRE( Factorial(}\DecValTok{2}\NormalTok{) == }\DecValTok{2} \NormalTok{);}
    \NormalTok{REQUIRE( Factorial(}\DecValTok{3}\NormalTok{) == }\DecValTok{6} \NormalTok{);}
    \NormalTok{REQUIRE( Factorial(}\DecValTok{10}\NormalTok{) == }\DecValTok{3628800} \NormalTok{);}
\NormalTok{\}}
\end{Highlighting}
\end{Shaded}

Produces this output when run:

\begin{verbatim}
===============================================================================
All tests passed (4 assertions in 1 test case)

\end{verbatim}

\subsubsection{Principles}\label{principles}

So, typically we have

\begin{itemize}
\itemsep1pt\parskip0pt\parsep0pt
\item
  Some \texttt{\#include} to get test framework
\item
  Our code that we want to test
\item
  Then make some assertions
\end{itemize}

\subsubsection{Catch / GoogleTest}\label{catch-googletest}

For example, in \href{https://github.com/philsquared/Catch}{Catch}:

\begin{Shaded}
\begin{Highlighting}[]
    \CommentTok{// TEST_CASE(<unique test name>, <test case name>)}
    \NormalTok{TEST_CASE( }\StringTok{"Factorials are computed"}\NormalTok{, }\StringTok{"[factorial]"} \NormalTok{) \{}
        \NormalTok{REQUIRE( Factorial(}\DecValTok{2}\NormalTok{) == }\DecValTok{2} \NormalTok{);}
        \NormalTok{REQUIRE( Factorial(}\DecValTok{3}\NormalTok{) == }\DecValTok{6} \NormalTok{);}
    \NormalTok{\}}
\end{Highlighting}
\end{Shaded}

In \href{https://code.google.com/p/googletest/}{GoogleTest}:

\begin{Shaded}
\begin{Highlighting}[]
    \CommentTok{// TEST(<test case name>, <unique test name>)}
    \NormalTok{TEST(FactorialTest, HandlesPositiveInput) \{}
      \NormalTok{EXPECT_EQ(}\DecValTok{2}\NormalTok{, Factorial(}\DecValTok{2}\NormalTok{));}
      \NormalTok{EXPECT_EQ(}\DecValTok{6}\NormalTok{, Factorial(}\DecValTok{3}\NormalTok{));}
    \NormalTok{\}}
\end{Highlighting}
\end{Shaded}

all done via C++ macros.

\subsubsection{Tests That Fail}\label{tests-that-fail}

What about Factorial of zero? Adding

\begin{Shaded}
\begin{Highlighting}[]
    \NormalTok{REQUIRE( Factorial(}\DecValTok{0}\NormalTok{) == }\DecValTok{1} \NormalTok{);}
\end{Highlighting}
\end{Shaded}

Produces something like:

\begin{Shaded}
\begin{Highlighting}[]
    \NormalTok{factorial2.cc:}\DecValTok{9}\NormalTok{: FAILED:}
    \NormalTok{REQUIRE( Factorial(}\DecValTok{0}\NormalTok{) == }\DecValTok{1} \NormalTok{)}
    \NormalTok{with expansion:}
    \DecValTok{0} \NormalTok{== }\DecValTok{1}
\end{Highlighting}
\end{Shaded}

\subsubsection{Fix the Failing Test}\label{fix-the-failing-test}

Leading to:

\begin{Shaded}
\begin{Highlighting}[]
\OtherTok{#define CATCH_CONFIG_MAIN  }\CommentTok{// This tells Catch to provide a main() - only do this in one cpp file}
\OtherTok{#include "../catch/catch.hpp"}

\DataTypeTok{unsigned} \DataTypeTok{int} \NormalTok{Factorial( }\DataTypeTok{unsigned} \DataTypeTok{int} \NormalTok{number ) \{}
    \CommentTok{//return number <= 1 ? number : Factorial(number-1)*number;}
    \KeywordTok{return} \NormalTok{number > }\DecValTok{1} \NormalTok{? Factorial(number}\DecValTok{-1}\NormalTok{)*number : }\DecValTok{1}\NormalTok{;}
\NormalTok{\}}

\NormalTok{TEST_CASE( }\StringTok{"Factorials are computed"}\NormalTok{, }\StringTok{"[factorial]"} \NormalTok{) \{}
    \NormalTok{REQUIRE( Factorial(}\DecValTok{0}\NormalTok{) == }\DecValTok{1} \NormalTok{);}
    \NormalTok{REQUIRE( Factorial(}\DecValTok{1}\NormalTok{) == }\DecValTok{1} \NormalTok{);}
    \NormalTok{REQUIRE( Factorial(}\DecValTok{2}\NormalTok{) == }\DecValTok{2} \NormalTok{);}
    \NormalTok{REQUIRE( Factorial(}\DecValTok{3}\NormalTok{) == }\DecValTok{6} \NormalTok{);}
    \NormalTok{REQUIRE( Factorial(}\DecValTok{10}\NormalTok{) == }\DecValTok{3628800} \NormalTok{);}
\NormalTok{\}}
\end{Highlighting}
\end{Shaded}

which passes:

\begin{verbatim}
===============================================================================
All tests passed (5 assertions in 1 test case)

\end{verbatim}

\subsubsection{Test Macros}\label{test-macros}

Each framework has a variety of macros to test for failure.
\href{https://github.com/philsquared/Catch}{Catch} has:

\begin{Shaded}
\begin{Highlighting}[]
    \NormalTok{REQUIRE(expression); }\CommentTok{// stop if fail}
    \NormalTok{CHECK(expression);   }\CommentTok{// doesn't stop if fails}
\end{Highlighting}
\end{Shaded}

If an exception is thrown, it's caught, reported and counts as a
failure.

Examples:

\begin{Shaded}
\begin{Highlighting}[]
    \NormalTok{CHECK( str == }\StringTok{"string value"} \NormalTok{);}
    \NormalTok{CHECK( thisReturnsTrue() );}
    \NormalTok{REQUIRE( i == }\DecValTok{42} \NormalTok{);}
\end{Highlighting}
\end{Shaded}

Others:

\begin{Shaded}
\begin{Highlighting}[]
    \NormalTok{REQUIRE_FALSE( expression )}
    \NormalTok{CHECK_FALSE( expression )}
    \NormalTok{REQUIRE_THROWS( expression ) # Must }\KeywordTok{throw} \NormalTok{an exception}
    \NormalTok{CHECK_THROWS( expression ) # Must }\KeywordTok{throw} \NormalTok{an exception, }\KeywordTok{and} \KeywordTok{continue} \NormalTok{testing}
    \NormalTok{REQUIRE_THROWS_AS( expression, exception type )}
    \NormalTok{CHECK_THROWS_AS( expression, exception type )}
    \NormalTok{REQUIRE_NOTHROW( expression )}
    \NormalTok{CHECK_NOTHROW( expression )}
\end{Highlighting}
\end{Shaded}

\subsubsection{Testing for Failure}\label{testing-for-failure}

To re-iterate:

\begin{itemize}
\itemsep1pt\parskip0pt\parsep0pt
\item
  You should test failure cases

  \begin{itemize}
  \itemsep1pt\parskip0pt\parsep0pt
  \item
    Force a failure
  \item
    Check that exception is thrown
  \item
    If exception is thrown, test passes
  \item
    (Some people get confused, expecting test to fail)
  \end{itemize}
\item
  Examples

  \begin{itemize}
  \itemsep1pt\parskip0pt\parsep0pt
  \item
    Saving to invalid file name
  \item
    Negative numbers passed into double arguments
  \item
    Invalid Physical quantities (e.g. -300 Kelvin)
  \end{itemize}
\end{itemize}

\subsubsection{Setup/Tear Down}\label{setuptear-down}

\begin{itemize}
\itemsep1pt\parskip0pt\parsep0pt
\item
  Some tests require objects to exist in memory
\item
  These should be set up

  \begin{itemize}
  \itemsep1pt\parskip0pt\parsep0pt
  \item
    for each test
  \item
    for a group of tests
  \end{itemize}
\item
  Frameworks do differ in this regards
\end{itemize}

\subsubsection{Setup/Tear Down in Catch}\label{setuptear-down-in-catch}

Referring to the
\href{https://github.com/philsquared/Catch/blob/master/docs/tutorial.md}{Catch
Tutorial}:

\begin{Shaded}
\begin{Highlighting}[]
\NormalTok{TEST_CASE( }\StringTok{"vectors can be sized and resized"}\NormalTok{, }\StringTok{"[vector]"} \NormalTok{) \{}

    \NormalTok{std::vector<}\DataTypeTok{int}\NormalTok{> v( }\DecValTok{5} \NormalTok{);}

    \NormalTok{REQUIRE( v.size() == }\DecValTok{5} \NormalTok{);}
    \NormalTok{REQUIRE( v.capacity() >= }\DecValTok{5} \NormalTok{);}

    \NormalTok{SECTION( }\StringTok{"resizing bigger changes size and capacity"} \NormalTok{) \{}
        \NormalTok{v.resize( }\DecValTok{10} \NormalTok{);}

        \NormalTok{REQUIRE( v.size() == }\DecValTok{10} \NormalTok{);}
        \NormalTok{REQUIRE( v.capacity() >= }\DecValTok{10} \NormalTok{);}
    \NormalTok{\}}
    \NormalTok{SECTION( }\StringTok{"resizing smaller changes size but not capacity"} \NormalTok{) \{}
        \NormalTok{v.resize( }\DecValTok{0} \NormalTok{);}

        \NormalTok{REQUIRE( v.size() == }\DecValTok{0} \NormalTok{);}
        \NormalTok{REQUIRE( v.capacity() >= }\DecValTok{5} \NormalTok{);}
    \NormalTok{\}}
    \NormalTok{SECTION( }\StringTok{"reserving bigger changes capacity but not size"} \NormalTok{) \{}
        \NormalTok{v.reserve( }\DecValTok{10} \NormalTok{);}

        \NormalTok{REQUIRE( v.size() == }\DecValTok{5} \NormalTok{);}
        \NormalTok{REQUIRE( v.capacity() >= }\DecValTok{10} \NormalTok{);}
    \NormalTok{\}}
    \NormalTok{SECTION( }\StringTok{"reserving smaller does not change size or capacity"} \NormalTok{) \{}
        \NormalTok{v.reserve( }\DecValTok{0} \NormalTok{);}

        \NormalTok{REQUIRE( v.size() == }\DecValTok{5} \NormalTok{);}
        \NormalTok{REQUIRE( v.capacity() >= }\DecValTok{5} \NormalTok{);}
    \NormalTok{\}}
\NormalTok{\}}
\end{Highlighting}
\end{Shaded}

So, Setup/Tear down is done before/after each section.

\subsection{Unit Testing Tips}\label{unit-testing-tips}

\subsubsection{C++ design}\label{c-design}

\begin{itemize}
\itemsep1pt\parskip0pt\parsep0pt
\item
  Stuff from above applies to Classes / Functions
\item
  Think about arguments:

  \begin{itemize}
  \itemsep1pt\parskip0pt\parsep0pt
  \item
    Code should be hard to use incorrectly.
  \item
    Use \texttt{const}, \texttt{unsigned} etc.
  \item
    Testing forces you to sort these out.
  \end{itemize}
\end{itemize}

\subsubsection{Test Driven Development
(TDD)}\label{test-driven-development-tdd}

\begin{itemize}
\itemsep1pt\parskip0pt\parsep0pt
\item
  Methodology

  \begin{enumerate}
  \def\labelenumi{\arabic{enumi}.}
  \itemsep1pt\parskip0pt\parsep0pt
  \item
    Write a test
  \item
    Run test, should fail
  \item
    Implement/Debug functionality
  \item
    Run test

    \begin{enumerate}
    \def\labelenumii{\arabic{enumii}.}
    \itemsep1pt\parskip0pt\parsep0pt
    \item
      if succeed goto 5
    \item
      else goto 3
    \end{enumerate}
  \item
    Refactor to tidy up
  \end{enumerate}
\end{itemize}

\subsubsection{TDD in practice}\label{tdd-in-practice}

\begin{itemize}
\itemsep1pt\parskip0pt\parsep0pt
\item
  Aim to get good coverage
\item
  Some people quote 70\% or more
\item
  What are the downsides?
\item
  Don't write `brittle' tests
\end{itemize}

\subsubsection{Behaviour Driven Development
(BDD)}\label{behaviour-driven-development-bdd}

\begin{itemize}
\itemsep1pt\parskip0pt\parsep0pt
\item
  Behaviour Driven Development (BDD)

  \begin{itemize}
  \itemsep1pt\parskip0pt\parsep0pt
  \item
    Refers to a
    \href{https://en.wikipedia.org/wiki/Behavior-driven_development}{whole
    area} of software engineering
  \item
    With associated tools and practices
  \item
    Think about end-user perspective
  \item
    Think about the desired behaviour not the implementation
  \item
    See
    \href{https://codeutopia.net/blog/2015/03/01/unit-testing-tdd-and-bdd/}{Jani
    Hartikainen} article.
  \end{itemize}
\end{itemize}

\subsubsection{TDD Vs BDD}\label{tdd-vs-bdd}

\begin{itemize}
\itemsep1pt\parskip0pt\parsep0pt
\item
  TDD

  \begin{itemize}
  \itemsep1pt\parskip0pt\parsep0pt
  \item
    Test/Design based on methods available
  \item
    Often ties in implementation details
  \end{itemize}
\item
  BDD

  \begin{itemize}
  \itemsep1pt\parskip0pt\parsep0pt
  \item
    Test/Design based on behaviour\\
  \item
    Code to interfaces (later in course)
  \end{itemize}
\item
  Subtly different
\item
  Aim for BDD
\end{itemize}

\subsubsection{Anti-Pattern 1:
Setters/Getters}\label{anti-pattern-1-settersgetters}

Testing every Setter/Getter.

Consider:

\begin{Shaded}
\begin{Highlighting}[]
   \KeywordTok{class} \NormalTok{Atom \{}

     \KeywordTok{public}\NormalTok{:}
       \DataTypeTok{void} \NormalTok{SetAtomicNumber(}\DataTypeTok{const} \DataTypeTok{int}\NormalTok{& number) \{ m_AtomicNumber = number; \}}
       \DataTypeTok{int} \NormalTok{GetAtomicNumber() }\DataTypeTok{const} \NormalTok{\{ }\KeywordTok{return} \NormalTok{m_AtomicNumber; \}}
       \DataTypeTok{void} \NormalTok{SetName(}\DataTypeTok{const} \NormalTok{std::string& name) \{ m_Name = name; \}}
       \NormalTok{std::string GetName() }\DataTypeTok{const} \NormalTok{\{ }\KeywordTok{return} \NormalTok{m_Name; \}}
     \KeywordTok{private}\NormalTok{:}
       \DataTypeTok{int} \NormalTok{m_AtomicNumber;}
       \NormalTok{std::string m_Name;}
   \NormalTok{\};}
\end{Highlighting}
\end{Shaded}

and tests like:

\begin{Shaded}
\begin{Highlighting}[]
    \NormalTok{TEST_CASE( }\StringTok{"Testing Setters/Getters"}\NormalTok{, }\StringTok{"[Atom]"} \NormalTok{) \{}

        \NormalTok{Atom a;}

        \NormalTok{a.SetAtomicNumber(}\DecValTok{1}\NormalTok{);}
        \NormalTok{REQUIRE( a.GetAtomicNumber() == }\DecValTok{1}\NormalTok{);}
        \NormalTok{a.SetName(}\StringTok{"Hydrogen"}\NormalTok{);}
        \NormalTok{REQUIRE( a.GetName() == }\StringTok{"Hydrogen"}\NormalTok{);}
\end{Highlighting}
\end{Shaded}

\begin{itemize}
\itemsep1pt\parskip0pt\parsep0pt
\item
  It feels tedious
\item
  But you want good coverage
\item
  This often puts people off testing
\item
  It also produces ``brittle'', where 1 change breaks many things
\end{itemize}

\subsubsection{Anti-Pattern 1:
Suggestion.}\label{anti-pattern-1-suggestion.}

\begin{itemize}
\itemsep1pt\parskip0pt\parsep0pt
\item
  Focus on behaviour.

  \begin{itemize}
  \itemsep1pt\parskip0pt\parsep0pt
  \item
    What would end-user expect to be doing?
  \item
    How would end-user be using this class?
  \item
    Write tests that follow the use-case
  \item
    Gives a more logical grouping
  \item
    One test can cover \textgreater{} 1 function
  \item
    i.e.~move away from slavishly testing each function
  \end{itemize}
\item
  Minimise interface.

  \begin{itemize}
  \itemsep1pt\parskip0pt\parsep0pt
  \item
    Provide the bare number of methods
  \item
    Don't provide setters if you don't want them
  \item
    Don't provide getters unless the user needs something
  \item
    Less to test. Use documentation to describe why.
  \end{itemize}
\end{itemize}

\subsubsection{Anti-Pattern 2: Constructing Dependent
Classes}\label{anti-pattern-2-constructing-dependent-classes}

\begin{itemize}
\itemsep1pt\parskip0pt\parsep0pt
\item
  Sometimes, by necessity we test groups of classes
\item
  Or one class genuinely Has-A contained class
\item
  But the contained class is expensive, or could be changed in future
\end{itemize}

\subsubsection{Anti-Pattern 2:
Suggestion}\label{anti-pattern-2-suggestion}

\begin{itemize}
\itemsep1pt\parskip0pt\parsep0pt
\item
  Read up on
  \href{https://martinfowler.com/articles/injection.html}{Dependency
  Injection}
\item
  Enables you to create and inject dummy test classes
\item
  So, testing again used to break down design, and increase flexibility
\end{itemize}

\subsubsection{Summary BDD Vs TDD}\label{summary-bdd-vs-tdd}

Aim to write:

\begin{itemize}
\itemsep1pt\parskip0pt\parsep0pt
\item
  Most concise description of requirements as unit tests
\item
  Smallest amount of code to pass tests
\item
  \ldots{} i.e.~based on behaviour
\end{itemize}

\subsection{Any Questions?}\label{any-questions}

\subsubsection{Homework - Overview}\label{homework---overview}

\begin{itemize}
\itemsep1pt\parskip0pt\parsep0pt
\item
  Example git repo, CMake, Catch template project:

  \begin{itemize}
  \itemsep1pt\parskip0pt\parsep0pt
  \item
    \href{https://github.com/MattClarkson/CMakeCatch2}{CMakeCatch2} -
    Simple
  \item
    \href{https://github.com/MattClarkson/CMakeCatchTemplate}{CMakeCatchTemplate}
    - Complex
  \end{itemize}
\item
  You should

  \begin{itemize}
  \itemsep1pt\parskip0pt\parsep0pt
  \item
    Clone, Build.
  \item
    Add unit test in Testing
  \item
    Run via ctest
  \item
    Find log file
  \end{itemize}
\end{itemize}

\subsubsection{Homework - 7}\label{homework---7}

\begin{itemize}
\item
  Imagine a simple function, e.g.~to add two numbers.
\item
  Play with unit tests until you understand the difference between:

\begin{Shaded}
\begin{Highlighting}[]
\DataTypeTok{int} \NormalTok{AddTwoNumbers(}\DataTypeTok{int} \NormalTok{a, }\DataTypeTok{int} \NormalTok{b);}
\DataTypeTok{int} \NormalTok{AddTwoNumbers(}\DataTypeTok{const} \DataTypeTok{int}\NormalTok{& a, }\DataTypeTok{const} \DataTypeTok{int}\NormalTok{&b);}
\DataTypeTok{void} \NormalTok{AddTwoNumbers(}\DataTypeTok{int}\NormalTok{* a, }\DataTypeTok{int}\NormalTok{*b, }\DataTypeTok{int}\NormalTok{* output);}
\DataTypeTok{void} \NormalTok{AddTwoNumbers(}\DataTypeTok{const} \DataTypeTok{int}\NormalTok{* }\DataTypeTok{const} \NormalTok{a, }\DataTypeTok{const} \DataTypeTok{int}\NormalTok{* }\DataTypeTok{const} \NormalTok{b);}
\end{Highlighting}
\end{Shaded}

  Now imagine, instead of integers, the variables all contained a large
  Image. Which type of function declaration would you use?
\end{itemize}

\subsubsection{Homework - 8}\label{homework---8}

\begin{itemize}
\itemsep1pt\parskip0pt\parsep0pt
\item
  Write a Fraction class
\item
  Write a print function to print nicely formatted fractions
\item
  Does the print function live inside or outside of the class?
\item
  Write a method \texttt{simplify()} which will simplify the fraction.
\item
  Unit test until you have at least got the hang of unit testing
\item
  Review your function arguments and return types
\end{itemize}
